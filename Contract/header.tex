\usepackage[T1]{fontenc} 

\usepackage[margin=2cm]{geometry} % [margin=1in]
% \geometry{
%     inner=4cm
% }


% uncomment for bernhard font
%\usepackage{bookman}
%\usepackage{cmbright}
%\usepackage[default]{opensans}

\usepackage{amsfonts}
\usepackage{amsmath}
\usepackage{amssymb}
\usepackage{cancel}
\usepackage{tikz}\usetikzlibrary{fit,arrows,shapes}
\usepackage{fancyhdr}
\usepackage[amsmath,hyperref]{ntheorem}
\usepackage{framed}
\usepackage{booktabs}
\usepackage{needspace}
\usepackage{diagbox}
\usepackage{parskip}
\usepackage{makecell}
\usepackage{varwidth}
\usepackage{multicol}
\usepackage{hyperref}
\usepackage{comment}
\usepackage[font=small,skip=5pt]{caption}
\usepackage{pgfplots}
\usepackage{xparse}
\usepackage{enumitem}
\usepackage{float}
\pgfplotsset{compat=newest}

%\setkomafont{title}{\LARGE\bfseries\rmfamily}
%\setkomafont{author}{\Large\mdseries\rmfamily}
%\setkomafont{section}{\Large\mdseries\rmfamily}
%\setkomafont{sectioning}{\rmfamily}
%\setkomafont{subsection}{\large\mdseries\rmfamily}
%\setkomafont{descriptionlabel}{\rmfamily}

\newenvironment{thinleftbar}{%
  \def\FrameCommand{\vrule width 1pt \hspace{10pt}}%
  \MakeFramed {\advance\hsize-\width \FrameRestore}}%
 {\endMakeFramed}

\newcommand{\thm}{
    \theoremstyle{plain}
    \theoremseparator{.}
    \theorembodyfont{\upshape}
    \theoremheaderfont{\bfseries}
    \theoremprework{\begin{thinleftbar}}
    \theorempostwork{\end{thinleftbar}}
}
\thm\newtheorem{definition}{Definition}[section]
\thm\newtheorem{proposition}[definition]{Proposition}
\thm\newtheorem{theorem}[definition]{Theorem}
\thm\newtheorem{lemma}[definition]{Lemma}
\thm\newtheorem{corollary}[definition]{Corollary}

\newcommand{\Grp}{\mathbb G}

\newcommand{\vv}[2]{
\left[
\begin{array}{c}
#1 \\ #2
\end{array}
\right]
}

\newcommand{\vvv}[3]{
\left[
\begin{array}{c}
#1 \\ #2 \\ #3
\end{array}
\right]
}

\newcommand{\vvvl}[3]{
[#1, #2, #3]^T
}

\newcommand{\m}[1]{
\left[\begin{array}{ccccc} #1 \end{array}\right]
}

\renewcommand{\vec}[1]{\overline{#1}}

\newcommand{\Exp}[1]{\mathbb{E}\left[#1\right]}

\renewcommand{\o}{\phantom{0}}

\newcommand{\hi}[1]{\textbf{#1}}

\setlength{\parskip}{6pt plus 2pt minus 2pt}
\setlength{\parindent}{0pt}

\DeclareMathOperator*{\argmax}{arg\,max}
\DeclareMathOperator*{\argmin}{arg\,min}

\newenvironment{diamondsec}{\noindent $\diamond$ \footnotesize}{}

\excludecomment{exercise}
%\newenvironment{exercise}{\begin{framed}\noindent\textbf{Exercise. }}{\end{framed}}

\newenvironment{example}{\noindent\textbf{Example. }}{}
\newenvironment{proof}{\noindent\textit{Proof. }}{\hfill$\square$}

\newcommand{\setuplecture}[2]{
    \lhead{\textsc{#1}}
    \rhead{Coursework Report}
    \title{\vspace{-2cm}\textsc{\textbf{#1}}\\\Large{\textbf{Coursework Report}}}
    \author{#2}
    \date{}
}

\newcommand{\titlestuff}{
    \maketitle

    % \thispagestyle{fancy}

    %\usefont{T1}{fos}{l}{n}\selectfont
}

% \pagestyle{fancy}
% \lhead{COMS20002 / CoCoNuT}
\cfoot{\thepage}


\tikzset{
    >=stealth',
    squig/.style={decorate,decoration={snake,post length=1mm}},
    brace/.style={decorate,decoration={brace,amplitude=5pt}},
    place/.style={draw,circle,inner sep=0.7em},
    token/.style={draw,circle,inner sep=0.3em,fill=gray},
    transition/.style 2 args={draw,rectangle,thick,minimum width=#1,minimum height=#2},
    rec/.style 2 args={draw,rectangle,minimum width=#1,minimum height=#2},
    cham/.style 2 args={draw,chamfered rectangle,minimum width=#1,minimum height=#2},
    rrec/.style 2 args={draw,rectangle,minimum width=#1,minimum height=#2,rounded corners=0.5mm},
    diam/.style 2 args={draw,diamond,minimum width=#1,minimum height=#2,aspect=1.5}
}

\hypersetup{
    colorlinks=true,
    bookmarksopen=true,
    allcolors=red
}
